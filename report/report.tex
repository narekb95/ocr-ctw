
\documentclass[a4paper,UKenglish,cleveref, autoref, thm-restate]{lipics-v2021}
%This is a template for producing LIPIcs articles. 
%See lipics-v2021-authors-guidelines.pdf for further information.
%for A4 paper format use option "a4paper", for US-letter use option "letterpaper"
%for british hyphenation rules use option "UKenglish", for american hyphenation rules use option "USenglish"
%for section-numbered lemmas etc., use "numberwithinsect"
%for enabling cleveref support, use "cleveref"
%for enabling autoref support, use "autoref"
%for anonymousing the authors (e.g. for double-blind review), add "anonymous"
%for enabling thm-restate support, use "thm-restate"
%for enabling a two-column layout for the author/affilation part (only applicable for > 6 authors), use "authorcolumns"
%for producing a PDF according the PDF/A standard, add "pdfa"

%\pdfoutput=1 %uncomment to ensure pdflatex processing (mandatatory e.g. to submit to arXiv)
%\hideLIPIcs  %uncomment to remove references to LIPIcs series (logo, DOI, ...), e.g. when preparing a pre-final version to be uploaded to arXiv or another public repository

%\graphicspath{{./graphics/}}%helpful if your graphic files are in another directory

\bibliographystyle{plainurl}% the mandatory bibstyle

\title{Dummy title} %TODO Please add

%\titlerunning{Dummy short title} %TODO optional, please use if title is longer than one line

\author{Narek Bojikian}{Humboldt University of Berlin, Germany}{bojikian@informatik.hu-berlin.de}{[todo]}{}%TODO mandatory, please use full name; only 1 author per \author macro; first two parameters are mandatory, other parameters can be empty. Please provide at least the name of the affiliation and the country. The full address is optional. Use additional curly braces to indicate the correct name splitting when the last name consists of multiple name parts.

\authorrunning{N. Bojikian} %TODO mandatory. First: Use abbreviated first/middle names. Second (only in severe cases): Use first author plus 'et al.'

\Copyright{Narek Bojikian} %TODO mandatory, please use full first names. LIPIcs license is "CC-BY";  http://creativecommons.org/licenses/by/3.0/

\ccsdesc[100]{\textcolor{red}{Replace ccsdesc macro with valid one}} %TODO mandatory: Please choose ACM 2012 classifications from https://dl.acm.org/ccs/ccs_flat.cfm 

\keywords{PACE Challenge, cutwidth} %TODO mandatory; please add comma-separated list of keywords

\category{} %optional, e.g. invited paper

\relatedversion{} %optional, e.g. full version hosted on arXiv, HAL, or other respository/website
%\relatedversiondetails[linktext={opt. text shown instead of the URL}, cite=DBLP:books/mk/GrayR93]{Classification (e.g. Full Version, Extended Version, Previous Version}{URL to related version} %linktext and cite are optional

%\supplement{}%optional, e.g. related research data, source code, ... hosted on a repository like zenodo, figshare, GitHub, ...
%\supplementdetails[linktext={opt. text shown instead of the URL}, cite=DBLP:books/mk/GrayR93, subcategory={Description, Subcategory}, swhid={Software Heritage Identifier}]{General Classification (e.g. Software, Dataset, Model, ...)}{URL to related version} %linktext, cite, and subcategory are optional

%\funding{(Optional) general funding statement \dots}%optional, to capture a funding statement, which applies to all authors. Please enter author specific funding statements as fifth argument of the \author macro.

\acknowledgements{[TODO]}%optional

%\nolinenumbers %uncomment to disable line numbering



%Editor-only macros:: begin (do not touch as author)%%%%%%%%%%%%%%%%%%%%%%%%%%%%%%%%%%
\EventEditors{John Q. Open and Joan R. Access}
\EventNoEds{2}
\EventLongTitle{PACE2HALG - Solving the one-sided crossing minimimization problem parameterized by cutwidth}
\EventShortTitle{CVIT 2016}
\EventAcronym{CVIT}
\EventYear{2016}
\EventDate{December 24--27, 2016}
\EventLocation{Little Whinging, United Kingdom}
\EventLogo{}
\SeriesVolume{42}
\ArticleNo{23}
%%%%%%%%%%%%%%%%%%%%%%%%%%%%%%%%%%%%%%%%%%%%%%%%%%%%%%

\begin{document}

\maketitle

%TODO mandatory: add short abstract of the document
\begin{abstract}
    In this article, we describe `HALG2PACE', a solver for the one-sided crossing minimization problem on graphs given together with a low-cutwidth linear arrangement. 
    This solver was developed as part of the PACE challenge 2024 - parameterized track. The solver is based on a dynamic programming scheme over the given linear arrangement, and admits FPT running time with single-exponential dependence on the cutwidth of the given linear arrangement. The solver is implemented in C++ and meets the requirements presented by PACE challenge. The solver was submitted on optil.io under the user name `narekb95' and is available on github at \url{https://github.com/narekb95/ocr-ctw}
\end{abstract}
\section{Introduction}

\section{Preliminaries}
For a plane $P$, a graph embedding $\mu$
The one-sided crossing minimization problem is a well-studied problem in graph embeddings, defined as follows: Let $P$ be a plane, and let $\ell_1, \ell_2$ be two horizontal non-overlapping lines in $P$. For a graph $G=$, an embedding of $G$ into $P$ is a mapping $\mu$, that maps each vertex of $G$ to a distinct points in $P$, and each edge $\{u,v\}\in E$  . A straight-line embedding is an embedding of $G$ to $P$ such that each edge is drawn as as 
Given a bipartite graph $G = (V_1,V_2,E)$ together with a fixed ordering $\pi_1$ over $V_1$, asked is an ordering $\pi_2$ over $V_2$ that minimizes the number of crossings in an embedding of $G$ into two parallel non-overlapping lines, such taht $V_1$  minimizes the total number of crossings 

and a linear arrangement of its vertices, the goal is to find a permutation of the vertices that minimizes the number of crossings in the drawing of the graph. The problem is known to be NP-hard in general, but admits efficient algorithms when the linear arrangement has low cutwidth. In this article, we describe a solver for the one-sided crossing minimization problem, which is based on a dynamic programming scheme over the given linear arrangement. The solver has a single exponential running time and is implemented in C++. The solver was submitted on optil.io under the user name `narekb95' and is available on github at 

PACE2HALG is a solver for the one-sided crossing minimization problem, for graphs given together with a low-cutwidth linear arrangements. It is based on a dynamic programming scheme over the given linear arrangement, and has a single exponential running time. The solver is implemented in C++ and uses the PACE challenge API. The solver was submitted on optil.io under the user name `narekb95' and is available on github at \url{https://github.com/narekb95/ocr-ctw}.

\end{document}
