
\documentclass[a4paper,UKenglish,cleveref, autoref, thm-restate]{lipics-v2021}
%This is a template for producing LIPIcs articles. 
%See lipics-v2021-authors-guidelines.pdf for further information.
%for A4 paper format use option "a4paper", for US-letter use option "letterpaper"
%for british hyphenation rules use option "UKenglish", for american hyphenation rules use option "USenglish"
%for section-numbered lemmas etc., use "numberwithinsect"
%for enabling cleveref support, use "cleveref"
%for enabling autoref support, use "autoref"
%for anonymousing the authors (e.g. for double-blind review), add "anonymous"
%for enabling thm-restate support, use "thm-restate"
%for enabling a two-column layout for the author/affilation part (only applicable for > 6 authors), use "authorcolumns"
%for producing a PDF according the PDF/A standard, add "pdfa"

%\pdfoutput=1 %uncomment to ensure pdflatex processing (mandatatory e.g. to submit to arXiv)
%\hideLIPIcs  %uncomment to remove references to LIPIcs series (logo, DOI, ...), e.g. when preparing a pre-final version to be uploaded to arXiv or another public repository

%\graphicspath{{./graphics/}}%helpful if your graphic files are in another directory

\bibliographystyle{plainurl}% the mandatory bibstyle

\title{Dummy title} %TODO Please add

%\titlerunning{Dummy short title} %TODO optional, please use if title is longer than one line

\author{Narek Bojikian}{Humboldt University of Berlin, Germany}{bojikian@informatik.hu-berlin.de}{[todo]}{}%TODO mandatory, please use full name; only 1 author per \author macro; first two parameters are mandatory, other parameters can be empty. Please provide at least the name of the affiliation and the country. The full address is optional. Use additional curly braces to indicate the correct name splitting when the last name consists of multiple name parts.

\authorrunning{N. Bojikian} %TODO mandatory. First: Use abbreviated first/middle names. Second (only in severe cases): Use first author plus 'et al.'

\Copyright{Narek Bojikian} %TODO mandatory, please use full first names. LIPIcs license is "CC-BY";  http://creativecommons.org/licenses/by/3.0/

\ccsdesc[100]{\textcolor{red}{Replace ccsdesc macro with valid one}} %TODO mandatory: Please choose ACM 2012 classifications from https://dl.acm.org/ccs/ccs_flat.cfm 

\keywords{PACE Challenge, cutwidth} %TODO mandatory; please add comma-separated list of keywords

\category{} %optional, e.g. invited paper

\relatedversion{} %optional, e.g. full version hosted on arXiv, HAL, or other respository/website
%\relatedversiondetails[linktext={opt. text shown instead of the URL}, cite=DBLP:books/mk/GrayR93]{Classification (e.g. Full Version, Extended Version, Previous Version}{URL to related version} %linktext and cite are optional

%\supplement{}%optional, e.g. related research data, source code, ... hosted on a repository like zenodo, figshare, GitHub, ...
%\supplementdetails[linktext={opt. text shown instead of the URL}, cite=DBLP:books/mk/GrayR93, subcategory={Description, Subcategory}, swhid={Software Heritage Identifier}]{General Classification (e.g. Software, Dataset, Model, ...)}{URL to related version} %linktext, cite, and subcategory are optional

%\funding{(Optional) general funding statement \dots}%optional, to capture a funding statement, which applies to all authors. Please enter author specific funding statements as fifth argument of the \author macro.

\acknowledgements{[TODO]}%optional

%\nolinenumbers %uncomment to disable line numbering



%Editor-only macros:: begin (do not touch as author)%%%%%%%%%%%%%%%%%%%%%%%%%%%%%%%%%%
\EventEditors{John Q. Open and Joan R. Access}
\EventNoEds{2}
\EventLongTitle{PACE2HALG - Solving the one-sided crossing minimimization problem parameterized by cutwidth}
\EventShortTitle{CVIT 2016}
\EventAcronym{CVIT}
\EventYear{2016}
\EventDate{December 24--27, 2016}
\EventLocation{Little Whinging, United Kingdom}
\EventLogo{}
\SeriesVolume{42}
\ArticleNo{23}
%%%%%%%%%%%%%%%%%%%%%%%%%%%%%%%%%%%%%%%%%%%%%%%%%%%%%%
\newtheorem{algorithm}{Algorithm}

\begin{document}

\maketitle

%TODO mandatory: add short abstract of the document
\begin{abstract}
    In this article, we describe `HALG2PACE', a solver for the one-sided crossing minimization problem on graphs given together with a low-cutwidth linear arrangement. 
    This solver was developed as part of the PACE challenge 2024 - parameterized track. The solver is based on a dynamic programming scheme over the given linear arrangement, and admits FPT running time with single-exponential dependence on the cutwidth of the given linear arrangement. The solver is implemented in C++ and meets the requirements presented by PACE challenge. The solver was submitted on optil.io under the user name `narekb95' and is available on github at \url{https://github.com/narekb95/ocr-ctw}
\end{abstract}
\section{Preliminaries}
For $n\in \mathbb{N}$, we denote by $[n]$ the set $\{1,\dots,n\}$.
A \emph{Permutation} $\pi$ of a set $S$ is a bijective mapping from $S$ to $[|S|]$. For $i\in[|S|]$ we denote by $\pi_i$ the element $\pi^-1(i)$ of $S$. Each permutation $\pi$ of a set corresponds (bijectively) to a linear ordering $\leq_{\pi}$, where for $u,v\in s$ it holds that $u\leq_{\pi} v$ if $\pi(u)\leq \pi(v)$. We will use these two terms interchangeably. In particular, we will call $\pi$ an ordering, when we mean the ordering $\leq_{\pi}$ underlying the permutation $\pi$.

A two-layered drawing of a bipartite graph $G=(A,B,E)$ is a drawing that maps its vertices into two different horizontal lines, such that the vertices of $V^1$ are mapped to one line, and the vertices of $V^2$ are mapped to the other. Edges are drawn as straight-line segments between the points corresponding to their endpoints. A two-layered is given as a mapping $\mu\colon V\rightarrow \mathbb{N}^2$. Implicitly, $\mu$ maps each edge $\{u,v\} \in E$ to the line segment between $\mu(u)$ and $\mu(v)$.
For $i\in[2]$, $V^1$, given an ordering $\pi$ over $V_i$, a two-layered drawing of $G$ respects $\pi$ if the order of the points corresponding to the images of the vertices of $V_i$ on the underlying horizontal line matches the order of the vertices themselves given by $\pi$.

In a two-layered drawing $\mu$, and for two edges $e_1, e_2 \in E$, we say that $e_1$ and $e_2$ cross in $\mu$, if $\mu(e_1)$ and $\mu{e_2}$ intersect in a point that is not an endpoint of either line segments. The \emph{number of crossings} of $\mu$ is the number of unordered pairs of edges $\{e_1, e_2\}\in \binom{E}{2}$ such that $e_1$ and $e_2$ cross in $\mu$.
The following observations are Folklore. We refer to \cite{DBLP:journals/algorithmica/DujmovicW04} for more information.
\begin{lemma}
    Given a bipartite graph $G=(V^1,V^2, E)$, together with two orderings $\pi_1, \pi_2$ over $V^1$ and $V^2$ respectively, the number of crossings of any two-layered drawing  of $G$ that respects $\pi_1$ and $\pi_2$ is invariant of the drawing itself, and determined solely by $\pi_1$ and $\pi_2$.
    We call this number of crossings of any such map as the \emph{crossings number of $(\pi_1, pi_2)$}.
\end{lemma}

\begin{definition}
    The \emph{one-sided crossing minimization problem} is defined as follows: Given a bipartite graph $G = (V^1,V^2,E)$ together with a fixed ordering $\pi_1$ over $V^1$, asked is a permutation $\pi_2$ over $V^2$ that minimizes the crossing number of $(\pi_1, \pi_2)$.
\end{definition}

A linear arrangement of a graph $G=(V,E)$ is linear ordering $\ell = v_1\leq \dots \leq v_n$ over the vertices of the graph. Let $V_i = \{v_1,\dots v_i\}$. For $v_i \in V$, we define the cut of $\ell$ at $v_i$ as the set of edges having one endpoint in $V_i$ and the other in $V\setminus V_i$, and call it the $i$th cut of $\ell$. The cut-graph $H_i = (L_i,R_i, E_i)$ is given by the edges of the $i$th cut together with their endpoints, where $L_i\subseteq V_i$ are the endpoints in $V_i$, and $R_i \subseteq V\setminus V_i$ are the other endpoints.


\section{The algorithm}
Finally, we introduce some notation for our algorithm. Along this work, we assume $G=(V^1,V^2, E)$ is the input graph, and $\ell$ a linear arrangement of $G$. Let $k$ be the largest size of a cut of $\ell$. Let $n = |V|$ and $m=|E|$.
Let $\pi_1$ be the give fixed ordering over $V^1$, and $\pi_2$ be the asked ordering.
Finally, for two different vertices $v_1, v_2 \in V^2$, we define $c(v_1,v_2)$ as
\[c(v_1,v_2) = |\{(w_2, w_1)\in N(v_2)\times N(v_1)\colon w_2 <_{\pi_1} w_1\}|.\] 
For two disjoint sets of vertices $X,X'\subseteq V_2$, we define
\[c(X,X') = \sum\limits_{(v_1,v_2)\in X\times X'}c(v_1,v_2).\]
Moreover, for a set $X\subseteq V^2$ we define $c(X)$ as the minimum crossing number of $(\pi_1, \pi')$ of $G[V_1, X]$ over all orderings $\pi'$ of $X$.

\begin{algorithm}\label{alg:subroutine}
    We start a subroutine to compute $c(X)$ for a small set $X$ in time $2^n\operatorname{poly}(n)$. The algorithm follows a similar approach to the well-known dynamic programming algorithm for the Hamiltonian Cycle given by Held and Karp \cite{DBLP:conf/acm/HeldK61} and independently by Bellman \cite{bellman1958combinatorial}. The subroutine iterates over all subsets of $X$ in increasing order (given by the subset relation), and computes an optimal ordering induced by each subset, by the recusive formula
    \[c(X) = min\{c(X\setminus\{v\}) + c(X\setminus \{v\}, v)\colon v \in X\}./\]    
\end{algorithm}

\begin{definition}
    Given a bipartite graph $G= (V_1, V_2, E)$ together with a permutation $\pi_1$ over $V_1$, a pair of vertices $(v_1,v_2)\in V_2^2$ is called suited (in respect to $\pi_1$), if $w_1\leq_{pi_1}w_2$ for each $w_1\in N_G(v_1)$ and $w_2\in N_G(v_2)$.
\end{definition}
We call a pair of vertices $v_1, v_2$ suited, if $c(v_1,v_2)\cdot c(v_2,v_1) = 0$.
Our algorithm is based on the following lemma:
\begin{lemma}[{{\cite[Fact 5]{DBLP:journals/algorithmica/DujmovicW04}}}]
    Let $u,v \in V^2$ be two suited vertices. In any optimal solution $\pi_2$, it holds that if $u\leq_{\pi_2} v$ then $c(u,v) = 0$.
\end{lemma}
Our algorithm follows a dynamic programming approach over the cuts of the linear arrangement $\ell$.
For $i\in[n]$, we define the set $S_i = V^2 \cap V(H_i)$ as the set of vertices of $V^2$ that are incident to edges of the $i$th cut.
The dynamic programming tables are indexed by subsets of $S_i$.
Formally, for $i\in [n]$ let $\mathcal{S}_i = \mathcal{P}(S_i)$ the power-set of $S_i$. we define the dynamic programming tables as vectors $T_i \in \mathbb{N}^{\mathcal{S}_i}$, where for $X\subseteq S_i$ we define $T_i[X]$ as follows:
\[T_i[X] = c(V_i\cup X) + c(V_i, V\setminus (V_i\cup X)).\]

At each cut $i$ our algorithm proceeds as follows: 
Let $S'_i = S_{i-1} \cup S_i$, and $F = S_{i-1} \setminus S_{i}$. We call $F_i$ the set of forget vertices at the $i$th cut. The algorithm first computes both $S'$ and $F$. Then for each subset $X \subseteq S'_i$ that is a super set of $X \supseteq F_i$, let $X' = X\setminus F_i$. The algorithm computes $T_{i}[X']$ as follows:
    \[T_i[X'] = \min\limits_{Y\subseteq X} c_1 + c2 + c3,\]
where
\begin{itemize}
    \item $c_1 = T_{i-1}(Y)$,
    \item $c_2 = c(X\setminus Y)$,
    \item $c_3 = c(Y, X)$,
    \item and $c_4 = c(F, S_i \setminus X')$.
\end{itemize}
Intuitively, the algorithm fixes consider orderings with $V_{i-1} \cup Y$ as a prefix, and append $X \setminus Y$ to this prefix (using the best ordering for $X\setminus Y$). Now we explain each summand in the states sum:
\begin{itemize}
    \item $c_1$ is the number of crossings induced by the prefix $V_{i-1}\cup Y$.
    \item $c_2$ is the number of crossings in induced by $X\setminus Y$ and is computing using \cref{alg:subroutine}.
    \item $c_3$ is the number of crossings introduced by appending $X$ to the prefix ordering (edges between $V_{i-1}$ and $X$ are accounted for in $c_1$).
    \item $c_4$ is the number of edges between $F$ and $V\setminus V_{i}$, since $F = V_i \setminus V_{i-1}$.
\end{itemize}

\section{Implementation details}
We use bit-masks to represent sets. Since each cut-edge has at most one endpoint in $V_1$, ist holds that $|S_i|\leq k$. To iterate over all supersets of $F$ that are subsets of $S'$ with constant time steps, we compute $S'_i\setminus F_i$, iterate over all its subsets $X'$ and compute $X = X' \cup F$ as the sets we are looking for.

In order to output the ordering we keep track of all sets $S_i$, and for each subset $X\subset S_i$ we keep track of the set $X\setminus Y$ that was appended to $V_i \cup Y$ to compute an optimal ordering at $X$. We use backtracking to generate each suffix at each step, and for each such suffix we use an additional call to \cref{alg:subroutine} to output an optimal ordering for this set.

\section{Correctness of the algorithm}
\bibliography{ref}
\end{document}
