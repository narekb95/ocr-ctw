
\documentclass[a4paper,UKenglish,cleveref, autoref, thm-restate]{lipics-v2021}
%This is a template for producing LIPIcs articles. 
%See lipics-v2021-authors-guidelines.pdf for further information.
%for A4 paper format use option "a4paper", for US-letter use option "letterpaper"
%for british hyphenation rules use option "UKenglish", for american hyphenation rules use option "USenglish"
%for section-numbered lemmas etc., use "numberwithinsect"
%for enabling cleveref support, use "cleveref"
%for enabling autoref support, use "autoref"
%for anonymousing the authors (e.g. for double-blind review), add "anonymous"
%for enabling thm-restate support, use "thm-restate"
%for enabling a two-column layout for the author/affilation part (only applicable for > 6 authors), use "authorcolumns"
%for producing a PDF according the PDF/A standard, add "pdfa"

%\pdfoutput=1 %uncomment to ensure pdflatex processing (mandatatory e.g. to submit to arXiv)
%\hideLIPIcs  %uncomment to remove references to LIPIcs series (logo, DOI, ...), e.g. when preparing a pre-final version to be uploaded to arXiv or another public repository

%\graphicspath{{./graphics/}}%helpful if your graphic files are in another directory

\bibliographystyle{plainurl}% the mandatory bibstyle

\title{H2PACE - Solving the One-Sided Crossing Minimization Problem Parameterize by Cutwidth} %TODO Please add

\titlerunning{H2PACE} %TODO optional, please use if title is longer than one line

\author{Narek Bojikian}{Humboldt University of Berlin, Germany}{bojikian@informatik.hu-berlin.de}{https://orcid.org/0000-0003-1072-4873}{}%TODO mandatory, please use full name; only 1 author per \author macro; first two parameters are mandatory, other parameters can be empty. Please provide at least the name of the affiliation and the country. The full address is optional. Use additional curly braces to indicate the correct name splitting when the last name consists of multiple name parts.

\authorrunning{N. Bojikian} %TODO mandatory. First: Use abbreviated first/middle names. Second (only in severe cases): Use first author plus 'et al.'

\Copyright{Narek Bojikian} %TODO mandatory, please use full first names. LIPIcs license is "CC-BY";  http://creativecommons.org/licenses/by/3.0/

\ccsdesc[500]{Theory of computation~Fixed parameter tractability}

\keywords{PACE Challenge, Cutwidth} %TODO mandatory; please add comma-separated list of keywords

\category{} %optional, e.g. invited paper

\relatedversion{} %optional, e.g. full version hosted on arXiv, HAL, or other respository/website
%\relatedversiondetails[linktext={opt. text shown instead of the URL}, cite=DBLP:books/mk/GrayR93]{Classification (e.g. Full Version, Extended Version, Previous Version}{URL to related version} %linktext and cite are optional

%\supplement{}%optional, e.g. related research data, source code, ... hosted on a repository like zenodo, figshare, GitHub, ...
%\supplementdetails[linktext={opt. text shown instead of the URL}, cite=DBLP:books/mk/GrayR93, subcategory={Description, Subcategory}, swhid={Software Heritage Identifier}]{General Classification (e.g. Software, Dataset, Model, ...)}{URL to related version} %linktext, cite, and subcategory are optional

%\funding{(Optional) general funding statement \dots}%optional, to capture a funding statement, which applies to all authors. Please enter author specific funding statements as fifth argument of the \author macro.

\acknowledgements{I would like to thank Sophia Heck for a very helpful discussion about this problem.}%optional

%\nolinenumbers %uncomment to disable line numbering



%Editor-only macros:: begin (do not touch as author)%%%%%%%%%%%%%%%%%%%%%%%%%%%%%%%%%%
\EventEditors{John Q. Open and Joan R. Access}
\EventNoEds{2}
\EventLongTitle{H2PACE - Solving the one-sided crossing minimimization problem parameterized by cutwidth}
\EventShortTitle{CVIT 2016}
\EventAcronym{CVIT}
\EventYear{2016}
\EventDate{December 24--27, 2016}
\EventLocation{Little Whinging, United Kingdom}
\EventLogo{}
\SeriesVolume{42}
\ArticleNo{23}
%%%%%%%%%%%%%%%%%%%%%%%%%%%%%%%%%%%%%%%%%%%%%%%%%%%%%%
\newtheorem{algorithm}{Algorithm}

\newcommand{\layerone}{\ensuremath{U}}
\newcommand{\layertwo}{\ensuremath{W}}

\begin{document}

\maketitle

%TODO mandatory: add short abstract of the document
\begin{abstract}
    We describe `H2PACE', a solver for the one-sided crossing minimization problem on graph given with a low-cutwidth linear arrangement.
    This solver was developed as part of the PACE challenge 2024 - parameterized track. The solver is based on a dynamic programming scheme over a given linear arrangement with running time $\mathcal{O}(3^{\operatorname{ctw}}\operatorname{poly}(n))$. The solver is implemented in C++ and meets the requirements presented by PACE challenge. The solver was submitted on optil.io under the username `narekb95' and is available on github at \url{https://github.com/narekb95/ocr-ctw} and under the digital object identifier \href{https://zenodo.org/records/12166628}{10.5281/zenodo.12166627}.
\end{abstract}
\section{Preliminaries}
For $n\in \mathbb{N}$, we denote by $[n]$ the set $\{1,\dots,n\}$.
A \emph{Permutation} $\pi$ of a set $S$ is a bijective mapping from $S$ to $[|S|]$. For $i\in[|S|]$ we denote by $\pi_i$ the element $\pi^{-1}(i)$ of $S$. Each permutation $\pi$ of a set corresponds (bijectively) to a linear ordering $\leq_{\pi}$, where for $u,v\in S$ it holds that $u\leq_{\pi} v$ if $\pi(u)\leq \pi(v)$. We will use these two terms interchangeably. In particular, we will call $\pi$ an ordering, when we mean the ordering $\leq_{\pi}$ underlying the permutation $\pi$.

Given a graph $G = (V,E)$ and a vertex $v$ of $G$, we denote by $N_G(v)$ the neighborhood of $v$ in $G$. We define $N_G[v] = N_G(v)\cup \{v\}$ as the closed neighborhood of $v$ in $G$. We omit the subscript $G$ when the graph is clear from the context.

A two-layered drawing of a bipartite graph $G=(A,B,E)$ is a drawing that maps its vertices into two different horizontal lines, such that the vertices of $\layerone$ are mapped to one line, and the vertices of $\layertwo$ are mapped to the other. Edges are drawn as straight-line segments between the points corresponding to their endpoints. A two-layered drawing is given as a mapping $\mu\colon V\rightarrow \mathbb{R}^2$. Implicitly, $\mu$ maps each edge $\{u,v\} \in E$ to the line segment between $\mu(u)$ and $\mu(v)$.

Let $X\in \{\layerone, \layertwo\}$. Given an ordering $\pi$ over $X$, a two-layered drawing $\mu$ of $G$ respects $\pi$, if the order of the images of the vertices of $X$  on the underlying horizontal line matches the order of the vertices of $X$ themselves given by $\pi$, i.e.\ for $u,v \in X$, and for $\mu(u) = (x_u, y_0)$, and $\mu(v) = (x_v, y_0)$, it holds that $x_u \leq x_v$ if and only if $u\leq_{\pi} v$.

In a two-layered drawing $\mu$, and for two edges $e_1, e_2 \in E$, we say that $e_1$ and $e_2$ cross in $\mu$, if $\mu(e_1)$ and $\mu({e_2})$ intersect in a point that is not an endpoint of either line segment. The \emph{number of crossings} of $\mu$ is the number of unordered pairs of edges $\{e_1, e_2\}\in \binom{E}{2}$ such that $e_1$ and $e_2$ cross in $\mu$.
The following observations are Folklore. We refer to \cite{DBLP:journals/algorithmica/DujmovicW04} for further details.
\begin{lemma}
    Given a bipartite graph $G=(\layerone,\layertwo, E)$, together with two orderings $\pi_1, \pi_2$ over $\layerone$ and $\layertwo$ respectively, the number of crossings of any two-layered drawing  of $G$ that respects $\pi_1$ and $\pi_2$ is invariant of the drawing itself, and is determined solely by $\pi_1$ and $\pi_2$.
    We call the number of crossings of any such drawing as the \emph{crossing number of $(\pi_1, \pi_2)$}.
\end{lemma}

\begin{definition}
    The \emph{one-sided crossing minimization problem} is defined as follows: Given a bipartite graph $G = (\layerone,\layertwo,E)$ together with a fixed ordering $\pi_1$ over $\layerone$, asked is a permutation $\pi_2$ over $\layertwo$ that minimizes the crossing number of $(\pi_1, \pi_2)$.
\end{definition}

A linear arrangement of a graph $G=(V,E)$ is linear ordering $\ell = v_1, \dots, v_n$ over $V$. Let $V_i = \{v_1,\dots v_i\}$. For $v_i \in V$, we define the cut of $\ell$ at $v_i$ as the set of edges having one endpoint in $V_i$ and the other in $V\setminus V_i$, and call it the $i$th cut of $\ell$.
The cutwidth of a linear arrangement $\ell$ is the largest size of a cut of $\ell$. For $\delta$ the largest degree of a graph, it holds that the cutwidth of any linear arrangement of this graph is at least $\lceil\delta/2\rceil$.
The cut-graph $H_i = (L_i,R_i, E_i)$ is given by the edges of the $i$th cut together with their endpoints, where $L_i\subseteq V_i$ are the endpoints in $V_i$, and $R_i \subseteq V\setminus V_i$ are the other endpoints.


\section{The algorithm}
We introduce some notation for our algorithm. Along this work, we assume $G=(\layerone,\layertwo, E)$ is the input graph, and $\ell$ a linear arrangement of $G$. Let $k$ be the largest size of a cut of $\ell$. Let $n = |V|$ and $m=|E|$.
Let $\pi_1$ be the given fixed ordering over $\layerone$, and $\pi_2$ be the asked ordering.
Finally, for two different vertices $v_1, v_2 \in \layertwo$, we define $c(v_1,v_2)$ as
\[c(v_1,v_2) = |\{(w_2, w_1)\in N(v_2)\times N(v_1)\colon w_2 <_{\pi_1} w_1\}|.\] 
Intuitively, $c(v_1,v_2)$ is the number of pairs of edges $e_1, e_2$ having endpoints in $v_1$ and $v_2$, that cross when we fix $\pi_2(v_1) < \pi_2(v_2)$. The following observation follows by a counting argument.
\begin{observation}
    It holds that $c(v_1,v_2)+ c(v_2,v_1) + |N(v_1)\cap N(v_2)| = |N(v_1)|\cdot |N(v_2)|.$
\end{observation}
For two disjoint sets of vertices $X,X'\subseteq W$, we define
$c(X,X') = \sum_{(v_1,v_2)\in X\times X'}c(v_1,v_2)$.
Moreover, for a set $X\subseteq \layertwo$ we define $c(X)$ as the minimum crossing number of $(\pi_1, \pi')$ in $G[V_1, X]$ over all orderings $\pi'$ of $X$.

\begin{algorithm}\label{alg:subroutine}
    We provide a subroutine to compute $c(X)$ for a small set $X$ in time $2^{|X|}\operatorname{poly}(|X|+k)$. The algorithm follows a similar approach to the well-known dynamic programming algorithm for the \textsc{Hamiltonian Cycle} problem given by Held and Karp \cite{DBLP:conf/acm/HeldK61} and independently by Bellman \cite{bellman1958combinatorial}. The subroutine iterates over all subsets of $X$ in an increasing order (given by the subset relation), and computes an optimal ordering induced by each subset using the recursive formula
    \[c(X) = min\{c(X\setminus\{v\}) + c(X\setminus \{v\}, v)\colon v \in X\}.\]    
\end{algorithm}

We call a pair of vertices $v_1, v_2 \in W$ \emph{suited}, if $c(v_1,v_2)\cdot c(v_2,v_1) = 0$. Our algorithm is based on the following lemma:

\begin{lemma}[{{\cite[Fact 5]{DBLP:journals/algorithmica/DujmovicW04}}}]\label{lem:suited}
    Let $u,v \in \layertwo$ be two suited vertices. In any optimal solution $\pi_2$, it holds that if $u\leq_{\pi_2} v$ then $c(u,v) = 0$.
\end{lemma}
Our algorithm follows a dynamic programming approach over the cuts of the linear arrangement $\ell$.
For $i\in[n]$, we define $\layertwo_i = V_i \cap \layertwo$, $A_i = \layertwo_i\setminus L_i$. Let $S_i = \layertwo \cap V(H_i)$ be the set of vertices of $\layertwo$ that are incident to edges of the $i$th cut.
The dynamic programming tables are indexed by subsets of $S_i$.
Formally, for $i\in [n]$ let $\mathcal{S}_i = \mathcal{P}(S_i)$ the power-set of $S_i$. we define the dynamic programming tables as vectors $T_i \in \mathbb{N}^{\mathcal{S}_i}$, where for $X\subseteq S_i$ it holds that:
\begin{equation} \label{eq:t-def}
T_i[X] = c(A_i\cup X) + c(A_i, V\setminus (A_i\cup X)).
\end{equation}
Intuitively, for each subset $X \subseteq S_i$ we fix $A_i\cup X$ as a prefix of the final ordering and count the optimal number of crossings in an optimal ordering of $A_i \cup X$ plus the number of crossings between $A_i$ and the rest of $S_i$. It follows by \cref{lem:suited} that the vertices of $A_i$ precede any vertex of $W\setminus (A_i\cup S_i)$ in an optimal ordering.

At each cut $i$ our algorithm proceeds as follows: 
Let $S'_i = S_{i-1} \cup S_i$, and $F_i = S_{i-1} \setminus S_{i}$. We call $F_i$ the set of forget vertices at the $i$th cut. The algorithm first computes both $S'$ and $F$. Then for each set $X$ with $F_i \subseteq X \subseteq S'_i$, i.e. $X$ is a subset of $S'_i$ and a super set of $F_i$, let $X' = X\setminus F_i$. The algorithm computes $T_{i}[X']$ as follows:
\begin{equation} \label{eq:t-rec}
    T_i[X'] = \min\limits_{Y\subseteq X}  T_{i-1}(Y) + c(X\setminus Y) + c(Y, X\setminus Y) + c(F, S_i \setminus X').
\end{equation}
Intuitively, the algorithm fixes an ordering with $A_{i-1} \cup Y$ as a prefix, and append $X \setminus Y$ to this prefix (using the best ordering for $X\setminus Y$ given by \cref{alg:subroutine}). The summands count the following crossings:
\begin{enumerate}
    \item The number of crossings induced by $A_{i-1}\cup Y$ and the crossings between $A_{i-1}$ and $V\setminus A_i$.
    \item The number of crossings induced by $X\setminus Y$.
    \item The number of crossings introduced by appending $Y$ to the prefix ordering.
    \item The number of edges between $F$ and $V\setminus (A_{i-1}\cup X)$, since $F = A_i \setminus A_{i-1}$.
\end{enumerate}

\section{Implementation details}

Our solver starts by removing isolated vertices from the graph, building a new graph $G'$. This ensures that each vertex has at least one neighbor. The solver sorts the adjacencies of each vertex by their order on the linear arrangement $\ell$ and assigns a range to each vertex, given by the first and the last index (in the linear arrangement) of a neighbor of this vertex. This allows to compute crossing numbers between two vertices in polynomial time in $k$.
After computing an optimal solution, the solver assigns to each vertex its original id, and appends isolated vertices in an arbitrary order.

We use bit-masks to represent sets. To enumerate all supersets of $F$ that are subsets of $S'$ with constant preprocessing and constant delay, we compute $S'_i\setminus F_i$, iterate over all its subsets $X'$ and compute $X = X' \cup F$ as the sets we are looking for.

In order to output the ordering, for each subset $X\subset S_i$ we keep track of the optimal suffix $X\setminus Y$ appended to $A_i \cup Y$. We use backtracking to generate each suffix in reverse order, and call \cref{alg:subroutine} to output an optimal ordering for this suffix.

\section{Sketch of Correctness}
The correctness follows from \cref{lem:suited}. By induction over the linear arrangement, one can show that the recursive formula for $T_i$ (\cref{eq:t-rec}) computes the right value given in \cref{eq:t-def}.

Since each edge has at most one endpoint in $\layertwo$, ist holds that $|S_i|\leq k$. The running time of the algorithm can be bounded by
\[
\sum_{i=1}^{n}\sum\limits_{S\subseteq S_i}\sum\limits_{X\subseteq S} \operatorname{poly}(k) = n3^k\operatorname{poly}(k).
\]
\bibliography{ref}
\end{document}
